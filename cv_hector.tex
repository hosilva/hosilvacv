%%%%%%%%%%%%%%%%%%%%%%%%%%%%%%%%%%%%%%%%%%%%%%%%%%%%%%%%%%%%%%%%%%%%%%%%
%%%%%%%%%%%%%%%%%%%%%% Simple LaTeX CV Template %%%%%%%%%%%%%%%%%%%%%%%%
%%%%%%%%%%%%%%%%%%%%%%%%%%%%%%%%%%%%%%%%%%%%%%%%%%%%%%%%%%%%%%%%%%%%%%%%

%%%%%%%%%%%%%%%%%%%%%%%%%%%%%%%%%%%%%%%%%%%%%%%%%%%%%%%%%%%%%%%%%%%%%%%%
%% NOTE: If you find that it says                                     %%
%%                                                                    %%
%%                           1 of ??                                  %%
%%                                                                    %%
%% at the bottom of your first page, this means that the AUX file     %%
%% was not available when you ran LaTeX on this source. Simply RERUN  %%
%% LaTeX to get the ``??'' replaced with the number of the last page  %%
%% of the document. The AUX file will be generated on the first run   %%
%% of LaTeX and used on the second run to fill in all of the          %%
%% references.                                                        %%
%%%%%%%%%%%%%%%%%%%%%%%%%%%%%%%%%%%%%%%%%%%%%%%%%%%%%%%%%%%%%%%%%%%%%%%%

%%%%%%%%%%%%%%%%%%%%%%%%%%%% Document Setup %%%%%%%%%%%%%%%%%%%%%%%%%%%%

% Don't like 10pt? Try 11pt or 12pt
\documentclass[10pt]{article}
\RequirePackage[T1]{fontenc}
\usepackage{amssymb}

% LaTeX will typeset using Computer Modern Roman, which a lot of
% non-mathematicians and non-engineers won't like. Also, a few PDF
% viewers may not render CMR very well. Instead, Times New Roman can
% be used. That's what this package does.
\usepackage{times}

% The automated optical recognition software used to digitize resume
% information works best with fonts that do not have serifs. This
% command uses a sans serif font throughout. Uncomment both lines (or at
% least the second) to restore a Roman font (i.e., a font with serifs).
% (NOTE: This requires the times package above)
%\renewcommand{\familydefault}{\sfdefault}

% This is a helpful package that puts math inside length specifications
\usepackage{calc}

% This package helps LaTeX auto-hyphenate hyphenated words if you use
% special hyphens. For example, bio\-/mimicry will properly hyphenate
% ``mimicry'' if necessary.
\usepackage[shortcuts]{extdash}

% Layout: Puts the section titles on left side of page
\reversemarginpar

%
%         PAPER SIZE, PAGE NUMBER, AND DOCUMENT LAYOUT NOTES:
%
% The next \usepackage line changes the layout for CV style section
% headings as marginal notes. It also sets up the paper size as either
% letter or A4. By default, letter was used. If A4 paper is desired,
% comment out the letterpaper lines and uncomment the a4paper lines.
%
% As you can see, the margin widths and section title widths can be
% easily adjusted.
%
% ALSO: Notice that the includefoot option can be commented OUT in order
% to put the PAGE NUMBER *IN* the bottom margin. This will make the
% effective text area larger.
%
% IF YOU WISH TO REMOVE THE ``of LASTPAGE'' next to each page number,
% see the note about the +LP and -LP lines below. Comment out the +LP
% and uncomment the -LP.
%
% IF YOU WISH TO REMOVE PAGE NUMBERS, be sure that the includefoot line
% is uncommented and ALSO uncomment the \pagestyle{empty} a few lines
% below.
%

%% Use these lines for letter-sized paper
\usepackage[paper=letterpaper,
            %includefoot, % Uncomment to put page number above margin
            marginparwidth=1.3in,     % Length of section titles
            marginparsep=.05in,       % Space between titles and text
            margin=0.8in,               % 1 inch margins
            includemp]{geometry}

% %% Use these lines for A4-sized paper
% \usepackage[paper=a4paper,
%            %includefoot, % Uncomment to put page number above margin
%            marginparwidth=28.5mm,    % Length of section titles
%            marginparsep=1.5mm,       % Space between titles and text
%            margin=25mm,              % 25mm margins
%            includemp]{geometry}

%% More layout: Get rid of indenting throughout entire document
\setlength{\parindent}{0in}

% Provides special list environments and macros to create new ones
\usepackage[shortlabels]{enumitem}

% Simpler bibsections for CV sections
% (thanks to natbib for inspiration)
%
% * For lists of references with hanging indents and no numbers:
%
%   \begin{bibsection}
%       \item ...
%   \end{bibsection}
%
% * For numbered lists of references (with hanging indents):
%
%   \begin{bibenum}
%       \item ...
%   \end{bibenum}
%
%   Note that bibenum numbers continuously throughout. To reset the
%   counter, use
%
%   \restartlist{bibenum}
%
%   at the place where you want the numbering to reset.

\makeatletter
\newlength{\bibhang}
\setlength{\bibhang}{1em}
\newlength{\bibsep}
 {\@listi \global\bibsep\itemsep \global\advance\bibsep by\parsep}
\newlist{bibsection}{itemize}{3}
\setlist[bibsection]{label=,leftmargin=\bibhang,%
        itemindent=-\bibhang,
        itemsep=\bibsep,parsep=\z@,partopsep=0pt,
        topsep=0pt}
\newlist{bibenum}{enumerate}{3}
\setlist[bibenum]{label=[\arabic*],resume,leftmargin={\bibhang+\widthof{[999]}},%
        itemindent=-\bibhang,
        itemsep=\bibsep,parsep=\z@,partopsep=0pt,
        topsep=0pt}
\let\oldendbibenum\endbibenum
\def\endbibenum{\oldendbibenum\vspace{-.6\baselineskip}}
\let\oldendbibsection\endbibsection
\def\endbibsection{\oldendbibsection\vspace{-.6\baselineskip}}
\makeatother

%% Reference the last page in the page number
%
% NOTE: comment the +LP line and uncomment the -LP line to have page
%       numbers without the ``of ##'' last page reference)
%
% NOTE: uncomment the \pagestyle{empty} line to get rid of all page
%       numbers (make sure includefoot is commented out above)
%
\usepackage{fancyhdr,lastpage}
\pagestyle{fancy}
%\pagestyle{empty}      % Uncomment this to get rid of page numbers
\fancyhf{}\renewcommand{\headrulewidth}{0pt}
\fancyfootoffset{\marginparsep+\marginparwidth}
\newlength{\footpageshift}
\setlength{\footpageshift}
          {0.5\textwidth+0.5\marginparsep+0.5\marginparwidth-2in}
\lfoot{\hspace{\footpageshift}%
       \parbox{4in}{\, \hfill %
                    {\it \arabic{page} of \protect\pageref*{LastPage}} % +LP
%                    \arabic{page}                               % -LP
                    \hfill \,}}

% Finally, give us PDF bookmarks
\usepackage{color,hyperref}
\usepackage[usenames,dvipsnames]{xcolor}
\definecolor{venetianred}{rgb}{0.78, 0.03, 0.08}
\hypersetup{colorlinks,breaklinks,
            linkcolor=Maroon,urlcolor=Plum,
            anchorcolor=Maroon,citecolor=Maroon}

\usepackage[anythingbreaks]{breakurl}

%%%%%%%%%%%%%%%%%%%%%%%% End Document Setup %%%%%%%%%%%%%%%%%%%%%%%%%%%%


%%%%%%%%%%%%%%%%%%%%%%%%%%% Helper Commands %%%%%%%%%%%%%%%%%%%%%%%%%%%%

%%% HEADING AT TOP OF CURRICULUM VITAE

% The title (name) with a horizontal rule under it
% (optional argument typesets an object right-justified across from name
%  as well)
%
% Usage: \makeheading{name}
%        OR
%        \makeheading[right_object]{name}
%
% Place at top of document. It should be the first thing.
% If ``right_object'' is provided in the square-braced optional
% argument, it will be right justified on the same line as ``name'' at
% the top of the CV. For example:
%
%       \makeheading[\emph{Curriculum vitae}]{Your Name}
%
% will put an emphasized ``Curriculum vitae'' at the top of the document
% as a title. Likewise, a picture could be included:
%
%   \makeheading[{\includegraphics[height=1.5in]{my_picture}}]{Your Name}
%
% the picture will be flush right across from the name. For this example
% to work, make sure the extra set of curly braces is included. Also
% makes ure that \usepackage{graphicx} is somewhere in the preamble.
\newcommand{\makeheading}[2][]%
        {\hspace*{-\marginparsep minus \marginparwidth}%
         \begin{minipage}[t]{\textwidth+\marginparwidth+\marginparsep}%
             {\large \bfseries #2 \hfill #1}\\[-0.15\baselineskip]%
                 \rule{\columnwidth}{1pt}%
         \end{minipage}}

%%% SECTION HEADINGS

% The section headings. Flush left in small caps down pseudo-margin.
%
% Usage: \section{section name}
\renewcommand{\section}[1]{\pagebreak[3]%
    \vspace{1.3\baselineskip}%
    \phantomsection\addcontentsline{toc}{section}{#1}%
    \noindent\llap{\scshape\smash{\parbox[t]{\marginparwidth}{\hyphenpenalty=10000\raggedright #1}}}%
    \vspace{-\baselineskip}\par}

%%% LISTS

% This macro alters a list by removing some of the space that follows the list
% (is used by lists below)
\newcommand*\fixendlist[1]{%
    \expandafter\let\csname preFixEndListend#1\expandafter\endcsname\csname end#1\endcsname
    \expandafter\def\csname end#1\endcsname{\csname preFixEndListend#1\endcsname\vspace{-0.6\baselineskip}}}

% These macros help ensure that items in outer-type lists do not get
% separated from the next line by a page break
% (they are used by lists below)
\let\originalItem\item
\newcommand*\fixouterlist[1]{%
    \expandafter\let\csname preFixOuterList#1\expandafter\endcsname\csname #1\endcsname
    \expandafter\def\csname #1\endcsname{\let\oldItem\item\def\item{\pagebreak[2]\oldItem}\csname preFixOuterList#1\endcsname}
    \expandafter\let\csname preFixOuterListend#1\expandafter\endcsname\csname end#1\endcsname
    \expandafter\def\csname end#1\endcsname{\let\item\oldItem\csname preFixOuterListend#1\endcsname}}
\newcommand*\fixinnerlist[1]{%
    \expandafter\let\csname preFixInnerList#1\expandafter\endcsname\csname #1\endcsname
    \expandafter\def\csname #1\endcsname{\let\oldItem\item\let\item\originalItem\csname preFixInnerList#1\endcsname}
    \expandafter\let\csname preFixInnerListend#1\expandafter\endcsname\csname end#1\endcsname
    \expandafter\def\csname end#1\endcsname{\csname preFixInnerListend#1\endcsname\let\item\oldItem}}

% An itemize-style list with lots of space between items
%
% Usage:
%   \begin{outerlist}
%       \item ...    % (or \item[] for no bullet)
%   \end{outerlist}
\newlist{outerlist}{itemize}{3}
    \setlist[outerlist]{label=\enskip\textbullet,leftmargin=*}
    \fixendlist{outerlist}
    \fixouterlist{outerlist}

% An environment IDENTICAL to outerlist that has better pre-list spacing
% when used as the first thing in a \section
%
% Usage:
%   \begin{lonelist}
%       \item ...    % (or \item[] for no bullet)
%   \end{lonelist}
\newlist{lonelist}{itemize}{3}
    \setlist[lonelist]{label=\enskip\textbullet,leftmargin=*,partopsep=0pt,topsep=0pt}
    \fixendlist{lonelist}
    \fixouterlist{lonelist}

% An itemize-style list with little space between items
%
% Usage:
%   \begin{innerlist}
%       \item ...    % (or \item[] for no bullet)
%   \end{innerlist}
\newlist{innerlist}{itemize}{3}
    \setlist[innerlist]{label=\enskip\textbullet,leftmargin=*,parsep=0pt,itemsep=0pt,topsep=0pt,partopsep=0pt}
    \fixinnerlist{innerlist}

% An environment IDENTICAL to innerlist that has better pre-list spacing
% when used as the first thing in a \section
%
% Usage:
%   \begin{loneinnerlist}
%       \item ...    % (or \item[] for no bullet)
%   \end{loneinnerlist}
\newlist{loneinnerlist}{itemize}{3}
    \setlist[loneinnerlist]{label=\enskip\textbullet,leftmargin=*,parsep=0pt,itemsep=0pt,topsep=0pt,partopsep=0pt}
    \fixendlist{loneinnerlist}
    \fixinnerlist{loneinnerlist}

%%% EXTRA SPACE

% To add some paragraph space between lines.
% This also tells LaTeX to preferably break a page on one of these gaps
% if there is a needed pagebreak nearby.
\newcommand{\blankline}{\quad\pagebreak[3]}
\newcommand{\halfblankline}{\quad\vspace{-0.5\baselineskip}\pagebreak[3]}

%%% FORMATTING MACROS

% Provides a linked \doi{#1} that links doi:#1 to http://dx.doi.org/#1
\usepackage{doi}
% To change the text before the DOI, adjust this command
%\renewcommand\doitext{doi:}

% Provides a linked \url{#1} that doesn't require escape characters
\usepackage{url}

% You can adjust the style \url{} uses here:
% (options are: same, rm, sf, tt; defaults to tt)
\urlstyle{same}

% For \email{ADDRESS}, links ADDRESS to the url mailto:ADDRESS
% (uncomment to typeset the e\-/mail address in typewriter font;
%  otherwise, will be typeset in the \urlstyle above)
%\DeclareUrlCommand\emaillink{\urlstyle{tt}}
\providecommand*\emaillink[1]{\nolinkurl{#1}}
\providecommand*\email[1]{\href{mailto:#1}{\emaillink{#1}}}

\providecommand\BibTeX{{B\kern-.05em{\sc i\kern-.025em b}\kern-.08em \TeX}}
\providecommand\Matlab{\textsc{Matlab}}

% Custom hyphenation rules for words that LaTeX has trouble with
% \hyphenation{bio-mim-ic-ry bio-in-spi-ra-tion re-us-a-ble pro-vid-er Media-Wiki}

\newcommand{\MYhref}[3][blue]{\href{#2}{\color{#1}{#3}}}

% To be published command
\newcommand{\tocome}[1]{{\textcolor{green}{{#1}} }}

%%%%%%%%%%%%%%%%%%%%%%%% End Helper Commands %%%%%%%%%%%%%%%%%%%%%%%%%%%

%%%%%%%%%%%%%%%%%%%%%%%%% Begin CV Document %%%%%%%%%%%%%%%%%%%%%%%%%%%%

\begin{document}
\makeheading{\sc Hector Okada da Silva}

\section{Personal and Contact Information}

% NOTE: Mind where the & separators and \\ breaks are in the following
%       table. Table is one row made up of three parboxes. The left
%       parbox has address info, the middle parbox has a vertical bar,
%       and the right parbox has phone and electronic contact
%       information.
%
% MACROS: \rcollength is the width of the right column of the table
%             (adjust it to your liking; default is 1.85in).
%         \spacewidth is width of area between left and right boxes.
%
\newlength{\rcollength}\setlength{\rcollength}{1.85in}%
\newlength{\spacewidth}\setlength{\spacewidth}{20pt}
%
\begin{tabular}[t]{@{}p{\textwidth-\rcollength-\spacewidth}@{}p{\spacewidth}@{}p{\rcollength}}%

% Address box
\parbox{\textwidth-\rcollength-\spacewidth}{%
{\it Name}: Hector Okada da Silva\\
{\it Citizenship}: Brazilian\\ \\
%
\href{http://www.illinois.edu}{University of Illinois at Urbana-Champaign}\\
% \href{http://www.montana.edu/xgi/}{eXtreme Gravity Institute}\\
\href{http://www.physics.illinois.edu}{Department of Physics}\\
Loomis Laboratory of Physics, Room 390Z\\
Urbana, IL 61801 USA}
&
% Uncomment to add a vertical bar in middle of contact information
% {\vrule width 0.5pt}
% \parbox[m][5\baselineskip]{\spacewidth}{} &

% Non-snail-mail contact information
\hspace{-1cm}\parbox{1.6\rcollength}{%
% \textit{Work:} +1-480-965-2899 \\
% \textit{Fax:} +1-480-965-2751 \\
% {E-mail:} \email{hector.oksilva@gmail.com}\\
{E-mail:} \email{hosilva@illinois.edu}\\
{Website:} \href{http://www.phy.olemiss.edu/~hosilva/}{www.phy.olemiss.edu/~hosilva}}

\end{tabular}

\section{Research Interests}

\textbf{Classical gravity:}
% and theoretical astrophysics:}
Classical aspects of relativistic gravity (general relativity and modified
theories). Structure, stability and dynamics of neutron stars and black holes,
including gravitational waves. Tests of gravity in the strong-field regime.
Equation of state of neutron stars.

\section{Employment}

\href{http://www.illinois.edu}{\textbf{University of Illinois at Urbana-Champaign}},
Urbana, IL, USA

\begin{outerlist}
\item[]
     {Department of Physics} \\
     August 2019 -- \emph{Ongoing}
        \begin{innerlist}
        \item Position: post-doctoral researcher
        \item Supervisor:
        \href{http://www.physics.montana.edu/people/faculty/yunes-nicolas.html}
       {Nicol\'as Yunes}
        \end{innerlist}
\end{outerlist}

\href{http://www.montana.edu}{\textbf{Montana State University}},
Bozeman, MT, USA

\begin{outerlist}

\item[]
     {Department of Physics} \\
     August 2017 -- August 2019
        \begin{innerlist}
        \item Position: post-doctoral researcher
        \item Supervisor:
        \href{http://www.physics.montana.edu/people/faculty/yunes-nicolas.html}
       {Nicol\'as Yunes}
        \end{innerlist}
\end{outerlist}

\section{Education}

\href{http://www.olemiss.edu/}{\textbf{University Mississippi}},
Oxford, MS, USA
\begin{outerlist}

\item[] PhD.,
        \href{http://olemiss.edu/depts/physics_and_astronomy/}
             {Department of Physics and Astronomy} \\
             May 2017
        \begin{innerlist}
        \item Title: \emph{Compact objects in relativistic theories of gravity}
        \item Supervisor:
              \href{http://www.phy.olemiss.edu/~berti/}
                   {Emanuele Berti}
        \end{innerlist}
\end{outerlist}

 \halfblankline

\href{http://www.ufpa.br/}{\textbf{Federal University of Par\'a}},
Bel\'em, PA, Brazil
\begin{outerlist}

\item[] M.Sc.,
        \href{http://www.ppgf.ufpa.br}
             {Graduate Program in Physics}, January 2011
        \begin{innerlist}
        \item Title: \emph{Dynamical Casimir effect in 1+1 dimensions}
        \item Supervisors:
        \href{http://lattes.cnpq.br/6642002445572793}{Danilo T. Alves}
        and
        \href{http://www.if.ufrj.br/docentes/carlos-farina-de-souza/}{Carlos Farina}
        \end{innerlist}

\item[] B.Sc.,
        \href{http://www.facfis.ufpa.br}
             {Department of Physics}, December 2008
        \begin{innerlist}
        \item Title: \emph{Exact solution for the energy density inside a non-stationary cavity with an arbitrary initial field state and applications}
        \item Supervisor:
        \href{http://lattes.cnpq.br/6642002445572793}{Danilo T. Alves}
        \end{innerlist}

\end{outerlist}

% % Add a little space to nudge next ``Ref'd Journal Publications'' marginpar
% % down to make room for tall ``Submitted Journal Publications''
% % marginpar. If there are enough submitted journal publications, this
% % space will not be needed (and should be removed).
% \vspace{0.1in}

\section{Awards, Honors \& Scholarships}
\restartlist{bibenum}

\begin{bibenum}
    \item \href{https://gwic.ligo.org/thesisprize/2017/}{\emph{The 2017 GWIC-Braccini Thesis Prize}},
        Gravitational Wave International Committee (2019).

	\item {\emph{NAOJ Visiting Joint Research}} travel grant,
	National Astronomical Observatory of Japan (2018).

    \item {\emph{Dissertation Fellowship} -- Spring 2017},
    University of Mississippi (2016).

    \item \href{http://physics.olemiss.edu/hector-okada-da-silva-is-a-winner-of-the-2016-graduate-student-achievement-award/}{\emph{Graduate Achievement Award}},
    University of Mississippi (2016).

    \item \href{http://physics.olemiss.edu/sigma-pi-sigma/}{\emph
    {Elected member of the Sigma Pi Sigma}}, the Physics Honor Society (2016).

    \item \emph{The Blue Apple Award} [For best student talk at the \href{http://www.phys.ufl.edu/events/gcgm8/index.html}{8th Gulf Coast
        Gravity Meeting}, University of Florida],
        % , Feb. 27--28, 2015],
    \href{https://www.aps.org/units/dgrav/}{American Physical Society Topical Group in Gravitation} (2015).

    \item Summer Research Assistantship,
    Graduate School, University of Mississippi (2015).

    \item University of Mississippi, Department of Physics and Astronomy,
    \emph{Zdravko Stipcevic Honors Fellowship}
    (2012 -- 2015).

    \item National Council for Scientific and Technological Development (CNPq) Scholarship , (2009 -- 2011).

    \item Foundation for Research Support of the State of Par\'a (FAPESPA) Scholarship,
    (2008 -- 2009).

    \item National Council for Scientific and Technological Development (CNPq) Scholarship , (2006 -- 2008).

\end{bibenum}

\section{Refereed Journal Publications}

	As of February 2019, my publications have gathered a total of 1252 citations and I have a
	\emph{h-index} of 16 according to the inSpire database. For the most up-to-date statistics
	see my \href{http://inspirehep.net/author/profile/H.O.Silva.1}{profile} in inSpire.

	\blankline

\restartlist{bibenum}
\begin{bibenum}

    \item D. T. Alves, E. R. Granhen, H. O. Silva and M. G. Lima,
    \emph{Quantum radiation force on the moving mirror of a cavity, with Dirichlet and Neumann boundary conditions for a vacuum, finite temperature, and a coherent state},
    \href{http://journals.aps.org/prd/abstract/10.1103/PhysRevD.81.025016}{Phys. Rev. D {\bf81} 025016 (2010)}.

    \item D. T. Alves, E. R. Granhen, H. O. Silva and M. G. Lima,
    \emph{Exact behavior of the energy density inside an one-dimensional oscillating cavity with a thermal state},
    \href{http://www.sciencedirect.com/science/article/pii/S0375960110009503}{Phys. Lett. A {\bf374} 3899-3907 (2010)}
    [\href{http://arxiv.org/abs/1002.2238}{arXiv:1002.2238}].

    \item H. O. Silva and C. Farina,
    \emph{Simple model for the dynamical Casimir effect for a static mirror with time-dependent properties},
    \href{http://journals.aps.org/prd/abstract/10.1103/PhysRevD.84.045003}{Phys. Rev. D {\bf84} 045003 (2011)}
    [\href{http://arxiv.org/abs/1102.2238}{arXiv:1102.2238}].
    \label{itm:robintd}

    \item A. L. C. Rego, H. O. Silva, D. T. Alves and C. Farina,
    \emph{New signatures of the dynamical Casimir effect in a superconducting circuit},
    \href{http://journals.aps.org/prd/abstract/10.1103/PhysRevD.90.025003}{Phys. Rev. D {\bf90} 025003 (2014)}
    [\href{http://arxiv.org/abs/1410.2511}{arXiv:1405.3720}].

    \item H. O. Silva, H. Sotani, E. Berti and M. Horbatsch,
    \emph{Torsional oscillations of neutron stars in scalar-tensor theory of gravity},
    \href{http://journals.aps.org/prd/abstract/10.1103/PhysRevD.90.124044}{Phys. Rev. D {\bf90} 124044 (2014)}
    [\href{http://arxiv.org/abs/1410.2511}{arXiv:1410.2511}].

    \item H. O. Silva, C. F. B. Macedo, E. Berti and L. C. B. Crispino,
    \emph{Slowly rotating anisotropic neutron stars in general relativity and scalar-tensor theory},
    \href{http://iopscience.iop.org/article/10.1088/0264-9381/32/14/145008/meta}{Class. Quantum Grav. {\bf32} 145008 (2015)}
    [\href{http://arxiv.org/abs/1411.6286}{arXiv:1411.6286}],
    (Chosen as
    \MYhref[venetianred]{http://iopscience.iop.org/collections}{\sc IOPSelect})
    % (Chosen as \href{http://iopscience.iop.org/collections}{\sc IOPSelect}).
    \label{itm:anisostar}

    \item K. Glampedakis, G. Pappas, H. O. Silva and E. Berti,
    \emph{Post-Tolman-Oppenheimer-Volkoff formalism for relativistic stars},
    \href{http://journals.aps.org/prd/abstract/10.1103/PhysRevD.92.024056}{Phys. Rev. D {\bf92} 024056 (2015)}
    [\href{http://arxiv.org/abs/1504.02455}{arXiv:1504.02455}].

    \item M. Horbatsch, H. O. Silva, D. Gerosa, P. Pani, E. Berti, L. Gualtieri
    and U. Sperhake,
    \emph{Tensor-multi-scalar theories: relativistic stars and 3+1 decomposition},
    \href{http://iopscience.iop.org/article/10.1088/0264-9381/32/20/204001/meta;jsessionid=575DFA8E53663683D64E28DF82CE0294.c1}{Class. Quantum Grav.
    {\bf32} 204001 (2015)}
    [\href{http://arxiv.org/abs/1505.07462}{arXiv:1505.07462}],
    (Chosen as
    \MYhref[venetianred]{http://iopscience.iop.org/collections}{\sc IOPSelect})
    \label{itm:multiscalar}

    \item A. Maselli, H. O. Silva, M. Minamitsuji and E. Berti,
    \emph{Slowly rotating black hole solutions in Horndeski gravity},
    \href{http://journals.aps.org/prd/abstract/10.1103/PhysRevD.92.104049}{Phys. Rev. D {\bf92} 104049 (2015)}
    [\href{http://arxiv.org/abs/1508.03044}{arXiv:1508.03044}].
    \label{itm:horndeski}

    \item H. O. Silva, H. Sotani and E. Berti,
    \emph{Low-mass neutron stars: universal relations, the nuclear symmetry energy and gravitational radiation},\\
    \href{http://mnras.oxfordjournals.org/content/459/4/4378}{MNRAS {\bf459}
    4378 (2016)}
    [\href{http://arxiv.org/abs/1601.03407}{arXiv:1601.03407}]

    \item A. Maselli, H. O. Silva, M. Minamitsuji and E. Berti,
    \emph{Neutron stars in Horndeski gravity},
    \href{http://journals.aps.org/prd/abstract/10.1103/PhysRevD.93.124056}{Phys. Rev. D {\bf93} 124056 (2016)}
    [\href{https://arxiv.org/abs/1603.04876}{arXiv:1603.04876}]

    \item M. Minamitsuji and H. O. Silva,
    \emph{Relativistic stars in scalar-tensor theories with disformal coupling},
    \href{http://journals.aps.org/prd/abstract/10.1103/PhysRevD.93.124056}{Phys. Rev. D {\bf93} 124041 (2016)}
    [\href{https://arxiv.org/abs/1604.07742}{arXiv:1604.07742}]
    \label{itm:disformal}

    \item K. Glampedakis, G. Pappas, H. O. Silva and E. Berti,
    \emph{Astrophysical application of the Post-Tolman-Oppenheimer-Volkoff formalism},
    \href{http://journals.aps.org/prd/abstract/10.1103/PhysRevD.94.044030}{Phys. Rev. D {\bf94} 044030 (2016)}
    [\href{https://arxiv.org/abs/1606.05106}{arXiv:1606.05106}]
    (Featured as {\color{venetianred}\sc Editors's Suggestion}).
    \label{itm:astroptov}

    \item K. Glampedakis, G. Pappas, H. O. Silva and E. Berti,
    \emph{Post-Kerr black hole spectroscopy},
    \href{https://journals.aps.org/prd/abstract/10.1103/PhysRevD.96.064054}{Phys. Rev. D {\bf96} 064054 (2017)}
    [\href{https://arxiv.org/abs/1706.07658}{arXiv:1706.07658}]
    \label{itm:postkerr}

    \item H. O. Silva and N. Yunes,
    \emph{I-Love-Q to the extreme},
    \href{http://iopscience.iop.org/article/10.1088/1361-6382/aa995a}{Class. Quantum Grav. {\bf35} 015005 (2017)}
    [\href{https://arxiv.org/abs/1710.00919}{arXiv:1710.00919}]

    \item J. Alsing, H. O. Silva and E. Berti,
    \emph{Evidence for a maximum mass cut-off in the neutron star mass distribution and constraints on the equation of state},
    \href{https://academic.oup.com/mnras/article-abstract/478/1/1377/4987228?redirectedFrom=fulltext}{MNRAS {\bf478} 1377 (2018)}
    [\href{https://arxiv.org/abs/1709.07889}{arXiv:1709.07889}]

    \item H. O. Silva, J. Sakstein, L. Gualtieri, T.~P. Sotiriou and E. Berti,
    \emph{Spontaneous scalarization of black holes and compact stars from a Gauss-Bonnet coupling},
    \href{https://journals.aps.org/prl/abstract/10.1103/PhysRevLett.120.131104}{Phys. Rev. Lett. {\bf120} 131104 (2018)}
    [\href{https://arxiv.org/abs/1711.02080}{arXiv:1711.02080}]
    \label{itm:edgb_sca}

    \item H. O. Silva and N. Yunes,
    \emph{Neutron star pulse profiles in scalar-tensor theories of gravity},
    \href{http://iopscience.iop.org/article/10.1088/1361-6382/aa995a}{Phys. Rev. D {\bf99} 044034 (2019)}
    [\href{https://arxiv.org/abs/1808.04391}{arXiv:1808.04391}]

    \item H. O. Silva, C. F. B. Macedo, T. P. Sotiriou, L. Gualtieri,
    	  J. Sakstein and E. Berti,
	\emph{Stability of scalarized black hole solutions in scalar-Gauss-Bonnet gravity},
	\href{https://journals.aps.org/prd/abstract/10.1103/PhysRevD.99.064011}{Phys. Rev. D {\bf99} 064011 (2019)}
	[\href{https://arxiv.org/abs/1812.05590}{arXiv:1812.05590}]

    \item C. F. B. Macedo, J. Sakstein, E. Berti, L. Gualtieri, H. O. Silva and T. P. Sotiriou,
    \emph{Self-interactions and Spontaneous Black Hole Scalarization}
    \href{https://journals.aps.org/prd/abstract/10.1103/PhysRevD.99.104041}{Phys. Rev. D {\bf 99}, 104041 (2019)}
    [\href{https://arxiv.org/abs/1903.06784}{arXiv:1903.06784}]

    \item H. O. Silva and N. Yunes,
    \emph{Neutron star x-ray burst oscillations as extreme gravity probes},
    \href{https://iopscience.iop.org/article/10.1088/1361-6382/ab3560}{Class. Quantum Grav. {\bf 36}, 17LT01 (2019)}
    [\href{https://arxiv.org/abs/1902.10269}{arXiv:1902.10269}]

    \item H. Sotani, H. O. Silva and G. Pappas,
    \emph{Finite size effects on the light curves of slowly-rotating neutron stars}
    \href{https://journals.aps.org/prd/abstract/10.1103/PhysRevD.100.043006}{Phys. Rev. D {\bf 100}, 043006 (2019)}
    [\href{https://arxiv.org/abs/1905.07668}{arXiv:1905.07668}]
    \label{itm:pp_fs}

    \item A. Saffer, H. O. Silva and N. Yunes,
    \emph{The exterior spacetime of relativistic stars in scalar-Gauss-Bonnet gravity}
    \href{https://journals.aps.org/prd/abstract/10.1103/PhysRevD.100.044030}{Phys. Rev. D {\bf 100}, 044030 (2019)}
    [\href{https://arxiv.org/abs/1903.07779}{arXiv:1903.07779}]

    \item K. Glampedakis and H. O. Silva,
    \emph{Eikonal quasinormal modes of black holes beyond general relativity}
    \href{https://journals.aps.org/prd/abstract/10.1103/PhysRevD.100.044040}{Phys. Rev. D {\bf 100}, 044040 (2019)}
    [\href{https://arxiv.org/abs/1906.05455}{arXiv:1906.05455}]

    \item H. O. Silva and N. Yunes,
    \emph{More than the sum of its parts: combining parameterized tests of extreme gravity}
    \href{https://link.aps.org/doi/10.1103/PhysRevD.100.084034}{Phys. Rev. D {\bf 100}, 084034 (2019)}
    [\href{https://arxiv.org/abs/1906.00485}{arXiv:1906.00485}]

    \item R. Nair, S. Perkins, H. O. Silva and N. Yunes,
    \emph{Fundamental physics implications on higher-curvature theories from
          the binary black hole signals in the LIGO-Virgo Catalog GWTC-1}
    \href{https://link.aps.org/doi/10.1103/PhysRevLett.123.191101}{Phys. Rev. Lett. {\bf 123}, 191101 (2019)},
    \href{https://journals.aps.org/prl/abstract/10.1103/PhysRevLett.124.169904}{Phys. Rev. Lett. {\bf 124}, 169904(E) (2020)}
    [\href{https://arxiv.org/abs/1905.00870}{arXiv:1905.00870}]
    \label{pub:nairPRL2020}

    \item H. O. Silva and M. Minamitsuji,
    \emph{Cosmological attractors to general relativity and spontaneous scalarization with disformal coupling}
    \href{https://link.aps.org/doi/10.1103/PhysRevD.100.104012}{Phys. Rev. D {\bf 100}, 104012 (2019)}
    [\href{https://arxiv.org/abs/1909.11756}{arXiv:1909.11756}]

    \item H. O. Silva and K. Glampedakis,
    \emph{Eikonal quasinormal modes of black holes beyond general relativity II: generalized scalar-tensor perturbations}
    \href{https://journals.aps.org/prd/abstract/10.1103/PhysRevD.101.044051}{Phys. Rev. D {\bf 101}, 044051 (2020)}
    [\href{https://arxiv.org/abs/1912.09286}{arXiv:1912.09286}]
\end{bibenum}

\blankline

\section{Publications Accepted or in Review}

\begin{bibenum}
    \item H. O. Silva, A. Miguel Holgado, Alejandro C\'{a}rdenas-Avenda\~{n}o and Nicol\'{a}s Yunes,
    \emph{Astrophysical and theoretical physics implications from multimessenger neutron star observations}
    [\href{https://arxiv.org/abs/2004.01253}{arXiv:2004.01253}]
\end{bibenum}

\blankline

\section{Review Papers}

\begin{bibenum}
    \item E. Berti, [{\it 47 authors}], H. O. Silva, [{\it 5 authors}],
    \emph{Testing general relativity with present and future astrophysical observations},
    \href{http://iopscience.iop.org/article/10.1088/0264-9381/32/24/243001}{Class. Quantum Grav. {\bf32} 243001 (2015)}
    [\href{http://arxiv.org/abs/1501.07274}{arXiv:1501.07274}].
\end{bibenum}

% \vspace{0.1in}
\blankline

\section{Publications in Conference Proceedings}

\begin{bibenum}

    \item D. T. Alves, E. R. Granhen, M. G. Lima, H. O. Silva and A. R. L. Rego,
    \emph{Time evolution of the energy density inside a one-dimensional non-static cavity with a vacuum, thermal and a coherent state},
    \href{http://iopscience.iop.org/article/10.1088/1742-6596/161/1/012032/meta}{J. Phys.: Conf. Ser. {\bf161} 012032 (2009)}
    [\href{http://arxiv.org/abs/0903.1305}{arXiv:0903.1305}]
    (Contribution to the proceedings of the
    \emph{60 years of the Casimir effect}
    conference).

    \item C. Farina, H. O. Silva, A. L. C. Rego and D. T. Alves,
    \emph{Time-dependent Robin boundary conditions in the dynamical Casimir effect},
    \href{http://www.worldscientific.com/doi/abs/10.1142/S2010194512007428}{Int. J. Mod. Phys. Conf. Ser. {\bf14} 306 (2012)}
    [\href{http://arxiv.org/abs/1201.3846}{arXiv:1201.3846}]
    (Contribution to the proceeding of the
    \emph{10th Quantum Field Theory Under the Influence of External Conditions}
    conference).

    \item H. O. Silva, A. Maselli, M. Minamitsuji and E. Berti,
    \emph{Compact objects in Horndeski gravity},
    \href{http://www.worldscientific.com/doi/abs/10.1142/S0218271816410066}{Int. J. Mod. Phys. D {\bf25} 1641006 (2016)}
    [\href{http://arxiv.org/abs/1602.05997}{arXiv:1602.05997}]
    (Contribution to the proceedings of the
    \emph{3rd Amazonian Symposium on Physics and 5th NRHEP Network Meeting}
    ).

\end{bibenum}

\blankline

\section{Book Chapters}
\restartlist{bibenum}
\begin{bibenum}
    \item A. L. C. Rego, D. T. Alves, E. R. Granhen, H. O. Silva, M. G. Lima and W. P. Pires,
    \emph{The Dynamical Casimir Effect}, in
    \emph{Trends in Physics - Festschrift in homage to Prof. Jos\'e
    Maria Filardo Bassalo}. Eds. M. S. D. Cattani, L. C. B. Crispino, M. O. C. Gomes and A. F. S. Santoro, (Editora Livraria da F\'isica, S\~ao Paulo, 2009).
\end{bibenum}

\blankline

\section{Media \& Press}
\restartlist{bibenum}
Some of my work has been featured in media and press:
\\
\begin{bibenum}

    \item Remya Nair, \href{https://physics.illinois.edu/research-highlights/article/36886}{``Testing Einstein when gravity waves''} \\
    Story the paper~\cite{pub:nairPRL2020}.
    July 2, 2020.

    \item Edwin B. Smith, \href{https://news.olemiss.edu/physics-alumnus-wins-international-award-gravitational-wave-thesis/}{``Physics Alumnus Wins International Award for Gravitational Wave Thesis''} \\
        Story on the 2017 GWIC-Braccini Thesis Award,
        June 25, 2019.

    \item Claire Fullerton, \href{http://cqgplus.com/2015/10/14/gravity-and-scalar-fields-live-long-and-prosper/}{``Gravity and scalar fields: live long and prosper?''} \\
        Invited contribution to \href{http://cqgplus.com}{\emph{CQG$+$}} based on publication \ref{itm:multiscalar},
        October 14, 2015.

    \item Rafael Rocha, \href{http://www.portal.ufpa.br/imprensa/noticia.php?cod=10864}{``Pesquisa em F\'isica ganha destaque em revista internacional''} \\
        Interview for the Federal University of Par\'a website on publication
        \ref{itm:anisostar},
        September, 2015.

    \item Claire Fullerton, \href{http://cqgplus.com/2015/09/23/spontaneous-scalarization-dead-or-alive/}{``Spontaneous scalarization: dead or alive?''} \\
        Invited contribution to \href{http://cqgplus.com}{\emph{CQG$+$}} based on publication \ref{itm:anisostar},
        September 23, 2015.
\end{bibenum}

\section{Conference, invited talks and
		colloquium presentations}

\restartlist{bibenum}
\begin{bibenum}
    \item H. O. Silva
    \emph{Neutron stars: equation of state and gravitational theory},
    \textcolor{Aquamarine}{invited} at the Nuclear Physics Journal Club, USA (2020).

    \item H. O. Silva
    \emph{Neutron stars as laboratories for fundamental physics},
    \textcolor{Aquamarine}{invited} departmental colloquium at \emph{Kent State University}
    in Ohio, USA (2020).

    \item H. O. Silva
    \emph{Scalar fields and compact objects}
    Astrophysics, Gravitation and Cosmology Seminar at \emph{University of Illinois at Urbana-Champaign},
    in Illinois, USA (2019).

    \item H. O. Silva
    \emph{Probing extreme gravity with x-ray burst oscillations}
    at the \emph{GR22/Amaldi13 Conference}
    in Valencia, Spain (2019).

    \item H. O. Silva
    \emph{Spontaneous black hole scalarization}
    at the \emph{GR22/Amaldi13 Conference}
    in Valencia, Spain (2019).

    \item H. O. Silva
    \emph{Parametrized tests of gravity: from stellar structure to gravitational waves}
    at the \emph{GR22/Amaldi13 Conference}
    in Valencia, Spain (2019).

    \item H. O. Silva
    \emph{Probing extreme gravity with NICER},
    \textcolor{Aquamarine}{invited} talk at the \emph{APS April Meeting 2019}
    in Denver, USA (2019).

    \item H. O. Silva
    \emph{The shape of rotating neutron stars and systematic errors
          in pulse profile observations parameter estimation}
    at the \emph{APS April Meeting 2019}
    in Denver, USA (2019).

    \item H. O. Silva,
    \emph{Scalar fields and strong-field gravity},
    \textcolor{Aquamarine}{invited} talk at the Institut d'Astrophysique de Paris, France (2019).

    \item H. O. Silva,
    \emph{Scalar fields and strong-field gravity},
    talk at \emph{RelAstro} at Montana State University, USA (2018).

	\item H. O. Silva,
    \emph{Scalar fields and strong-field gravity: spontaneous scalarization of compact objects},
    \textcolor{Aquamarine}{invited} talk at the Kavli Institute for Cosmological Physics, University of Chicago, USA (2019).

    \item H. O. Silva,
    \emph{Illuminating the strong-field regime of gravity},
    talk at \emph{RelAstro} at Montana State University, USA (2018).

	\item H. O. Silva,
    \emph{Illuminating the strong-field regime of gravity},
    \textcolor{Aquamarine}{invited} talk at the National Astronomical Observatory of Japan, Japan (2018).

    \item H. O. Silva,
    \emph{A f\'isica extrema da estrelas de n\^eutrons},
    \textcolor{Aquamarine}{invited} talk at Universidade Federal do Par\'a, Brazil (2018).

    \item H. O. Silva,
    \emph{Estrelas de n\^eutrons: laborat\'orios celestes para f\'isica fundamental},
    \textcolor{Aquamarine}{invited} talk at Universidade Federal do Par\'a, Brazil (2018).

    \item H. O. Silva,
    \emph{I-Love-Q to the extreme},
    talk at \emph{RelAstro} at Montana State University, USA (2018).

    \item H. O. Silva,
    \emph{Neutron star masses: from astro to fundamental physics},
    talk at \emph{RelAstro} at Montana State University, USA (2017).

    \item H. O. Silva,
    \emph{Probing the strong-field regime of gravity with neutrons stars},
    \textcolor{Aquamarine}{invited} talk at Montana State University, USA (2017).

    \item H. O. Silva, K. Glampedakis, G. Pappas and E. Berti
    \emph{Appplications of the post-Tolman-Oppenheimer-Volkoff formalism},
    at the \emph{APS April Meeting 2017}
    in Washington DC, USA (2017).

    \item H. O. Silva, M. Minamitsuji
    \emph{Relativistic stars in scalar-tensor theories with disformal coupling},
    at the \emph{APS April Meeting 2017}
    in Washington DC, USA (2017).

    \item H. O. Silva,
    \emph{Confronting scalar-tensor theories of gravity
    against binary-pulsar observations},
    at the \emph{II Physics Graduate Student Research Symposium}
    in Oxford, USA (2016).

    \item H. O. Silva,
    \emph{Neutron stars in scalar-tensor theories of gravity},
    \textcolor{Aquamarine}{invited} talk at Universidade de Lisboa
    in Lisbon, Portugal (2016).

    \item H. O. Silva,
    \emph{Neutron stars in scalar-tensor theories of gravity},
    \textcolor{Aquamarine}{invited} talk at Universidade de Aveiro
    in Aveiro, Portugal (2016).

    \item H. O. Silva,
    \emph{Neutron stars in scalar-tensor theories of gravity},
    \textcolor{Aquamarine}{invited} talk at Instituto Superior T\'ecnico
    in Lisbon, Portugal (2016).

    \item H. O. Silva,
    \emph{Low-mass neutron stars: universal relations, the nuclear symmetry energy and gravitational radiation}
    at the {APS April Meeting},
    in Salt Lake City, USA (2016).

    \item H. O. Silva,
    \emph{Neutron stars as strong gravity probes},
    \textcolor{Aquamarine}{invited} talk at Mississippi State University's Journal Club
    in Starkville, USA (2015).

    \item H. O. Silva, A. Maselli, M. Minamitsuji and E. Berti,
    \emph{Slowly rotating black hole solutions in Horndeski gravity},
    at the \emph{III Amazonian Symposium on Physics and V NRHEP Network Meeting}
    in Bel\'em, Brazil (2015).

    \item H. O. Silva, E. Berti, K. Glampedakis and G. Pappas,
    \emph{Testing general relativity with neutron stars: a new
    parametrized formalism},
    at the \emph{UM Research Day}
    in Oxford, USA (2015).

    \item H. O. Silva,
    \emph{No-hair theorems in Horndeski gravity},
    at the \emph{Physics Graduate Student Research Symposium}
    in Oxford, USA (2015).

    \item H. O. Silva, E. Berti, K. Glampedakis and G. Pappas,
    \emph{A post-TOV formalism for relativistic stars},
    at the \emph{IV NRHEP Network Meeting}
    in Rome, Italy (2015).

    \item H. O. Silva,
    \emph{Tests of strong gravity with neutron star},
    at the \emph{(Non-)universal properties of neutron stars}
    in Bremen, Germany (2015).

    \item H. O. Silva, E. Berti, K. Glampedakis and G. Pappas,
    \emph{A post-TOV formalism for relativistic stars},
    at the \emph{8th Gulf Coast Gravity Meeting}
    in Gainesville, USA (2015).

    \item C. Farina and H. O. Silva,
    \emph{Dynamical Casimir effect without moving mirrors},
    at the \emph{V Week of the Federal University of Par\'a Graduate Program in Physics}
    in Bel\'em, Brazil (2010).

    \item J. S. S. J\'unior and H. O. Silva,
    \emph{Zeta Functions and the Casimir effect},
    at the \emph{IV Week of the Federal University of Par\'a Graduate Program in Physics}
    in Bel\'em, Brazil (2009).

    \item H. O. Silva,
    \emph{Schwinger's method for calculating non-relativistic
    propagators in quantum mechanics},
    at the \emph{Journal Club}
    in Bel\'em, Brazil (2009).

    \item D. T. Alves, H. O. Silva and A. N. Braga,
    \emph{Quantum vacuum effects},
    at the \emph{Physics Freshman Week of Federal University of Par\'a}
    in Bel\'em, Brazil (2008).

    \item H. O. Silva and D. T. Alves,
    \emph{Quantum field theory in spaces with boundaries
    with emphasis on the dynamical Casimir effect},
    at the \emph{XIV Seminar of Undergraduate Research of
    Federal University of Par\'a}
    in Bel\'em, Brazil (2008).

    \item H. O. Silva and D. T. Alves,
    \emph{Quantum field theory in spaces of boundaries},
    at the \emph{XVIII Seminar of Undergraduate Research of
    Federal University of Par\'a}
    in Bel\'em, Brazil (2007).

    \item H. O. Silva and D. T. Alves,
    \emph{Quantum field theory in the presence of boundaries},
    at the \emph{XVII Seminar of Undergraduate Research of
    Federal University of Par\'a}
    in Bel\'em, Brazil (2006).
\end{bibenum}

\section{Conference Posters}
\restartlist{bibenum}

\begin{bibenum}

    \item K. Glampedakis, G. Pappas, H. O. Silva and E. Berti,
    \emph{A post-TOV formalism for relativistic stars},
    at the \emph{Compact Objects as Astrophysical and Gravitational Probes}
    in Leiden, Netherlands (2015).

    \item H. O. Silva, H. Sotani, E. Berti and M. Horbatsch,
    \emph{Torsional oscillations of neutron stars in scalar-tensor theory},
    at the \emph{GR@99}
    in Bad Honnef, Germany (2014).

    \item A. L. C. Rego, C. Farina, H. O. Silva and D. T. Alves,
    \emph{Dynamical Casimir effect with time dependent Robin
    boundary conditions in 3+1 dimensions},
    at the \emph{Physics Meeting}
    in Foz do Igua\c{c}u, Brazil (2010).

    \item J. S. S. J\'unior and H. O. Silva,
    \emph{Casimir effect with soft boundary conditions at finite temperature},
    at the \emph{II Amazonian School on Quantum Theory and Applications}
    in Bel\'em, Brazil (2010).

    \item J. S. S. J\'unior and H. O. Silva,
    \emph{Mass and temperature corrections to the Casimir effect},
    at the \emph{XXVII Meeting of Physicists from North and Northeast}
    in Bel\'em, Brazil (2009).

    \item D. T. Alves, E. R. Granhen, H. O. Silva and M. G. Lima,
    \emph{Quantum radiation force on the mirrors of a non-static cavity,
    with Dirichlet and Neumann boundary conditions for a vacuuum,
    finite Temperature and coherent state},
    at the \emph{Workshop on Quantum Nonstationary Systems}
    in Bras\'ilia, Brazil (2009).

    \item J. S. S. J\'unior, H. O. Silva and D. T. Alves,
    \emph{Casimir effect and lattice regularization},
    at the \emph{V Undergraduate Research Journey of the Program of
    Tutorial Education of Par\'a} in Bel\'em, Brazil (2009).

    \item H. O. Silva, C. F. B. Macedo and L. C. B. Crispino,
    \emph{Solution of the one-dimensional Schr\"odinger equation for a
    particle in an infinite potential well in a discrete spacetime},
    at the \emph{V Undergraduate Research Journey of the Program of
    Tutorial Education of Par\'a} in Bel\'em, Brazil (2009).

\end{bibenum}

\blankline

\section{Visits}
\restartlist{bibenum}

Long scientific visits:

\halfblankline

% \begin{bibsection}
\begin{bibenum}

	\item \href{https://www.nao.ac.jp/en/}{\textbf{National Astronomical Observatory of Japan}}
	\hfill{September -- October 2018}\\
	Visited the Division of Theoretical Astronomy of the National Astronomical Observatory of Japan
    and collaborated with Hajime Sotani. Visit resulted in the publication~\ref{itm:pp_fs}.\\
	Host: \href{http://inspirehep.net/author/profile/H.Sotani.1}{Hajime Sotani}.

	\item \href{http://www.ufpa.br}{\textbf{Universidade Federal do Par\'a}}
    \hfill{June 2018}\\
        Visited the Departamento de F\'isica and the Faculdade de Ci\^encias of the Universidade Federal do
        Par\'a, Bel\'em and Salin\'opolis campuses. \\
        Hosts: \href{http://inspirehep.net/author/profile/L.C.B.Crispino.1}{Luis C. B. Crispino} and
        \href{https://caiomacedo.weebly.com/}{Caio F. B. Macedo}

    \item \href{http://www.nottingham.ac.uk}{\textbf{University of Nottingham}}
    \hfill{May -- July 2017}\\
        Visited the University of Nottingham and collaborated with
        Prof. Thomas P. Sotiriou. Collaboration resulted in
        the publication~\ref{itm:edgb_sca}. \\
        Host: \href{http://thomassotiriou.wixsite.com/challenginggr}{Thomas P. Sotiriou}.

    \item \href{tecnico.ulisboa.pt}{\textbf{Instituto Superior T\'ecnico}}
    \hfill{April -- July 2016}\\
        Visited the Gravitation at T\'ecnico
        (\href{http://centra.tecnico.ulisboa.pt/network/grit/about/}{GRIT})
        group, led by Prof. Vitor Cardoso. Collaboration resulted in the
        publications \ref{itm:disformal} and \ref{itm:astroptov}.\\
        Host: \href{http://centra.ist.utl.pt/~vitor/}{Vitor Cardoso}.

    \item \href{tecnico.ulisboa.pt}{\textbf{Instituto Superior T\'ecnico}}
    \hfill{May -- July 2015}\\
        Visited the Gravitation at T\'ecnico
        (\href{http://centra.tecnico.ulisboa.pt/network/grit/about/}{GRIT})
        group, led by Prof. Vitor Cardoso. Collaboration resulted in the
        publication \ref{itm:horndeski}.\\
        Host: \href{http://centra.ist.utl.pt/~vitor/}{Vitor Cardoso}.

    \item \href{www.ufrj.br}{\textbf{Federal University of Rio de Janeiro}}
    \hfill{January -- July 2010}\\
        I was an exchange student visiting the
        \href{http://www.if.ufrj.br}{Physics Institute}.
        I attended a graduate level special topics course on quantum vacuum
        effects. Collaboration with the group of Prof. Carlos Farina
        resulted in the publication \ref{itm:robintd}. \\
        Host: \href{http://www.if.ufrj.br/docentes/carlos-farina-de-souza/}
        {Carlos Farina}. \\
\end{bibenum}

\section{Mentoring}
\restartlist{bibenum}

\begin{bibenum}
		\item \textbf{Reagan Cox}\\
        Undergraduate student in physics, Montana State University, USA. \\
        Project: \emph{Neutron stars in massive scalar-tensor gravity}. \\
        Primary adviser: Nicol\'as Yunes. \\
        Period: 2017--\emph{ongoing}.

    \item \textbf{Tan\'isia de F\'atima de Moraes Cardoso}\\
        Undergraduate student in physics, Federal University of Par\'a, Brazil. \\
        Project: \emph{Path integrals in quantum mechanics}. \\
        Primary adviser: Silvana Perez. \\
        Period: 2009--2010.

    \item \textbf{Jocivaldo Siqueira da Silva J\'unior}\\
        Undergraduate student in physics, Federal University of Par\'a, Brazil. \\
        Project: \emph{Casimir Effect: static and dynamical}. \\
        Primary adviser: Danilo T. Alves. \\
        Period: 2009--2011.

    \item \textbf{Monique Val\'erio Silva}\\
        High-school student, Federal University of Par\'a, Brazil \\
        Project: \emph{Studies on mechanics in high-school with the aid of computational techniques}. \\
        Primary adviser: Luis Carlos Bassalo Crispino. \\
        Period: 2009.
\end{bibenum}

\vspace{0.1in}

% \end{bibsection}

\section{Teaching Experience}

\href{http://www.olemiss.edu/}{\textbf{University of Mississippi}},
Oxford, MS, USA
\begin{outerlist}

\item[] \textit{Grader}%
        \hfill{August -- December 2015}
\begin{innerlist}
\item Graded assignments for graduate level course in general
relativity (PHYS~729).
\item Main instructor: \href{http://www.phy.olemiss.edu/~luca/}{Luca Bombelli}.
\end{innerlist}

\item[] \textit{Grader}%
        \hfill{August -- December 2015}
\begin{innerlist}
\item Graded assignments for introductory astronomy class (ASTR~103).
\item Main instructor: James Hill.
\end{innerlist}

\item[] \textit{Teaching Assistant} \hfill{August 2012 -- July 2014}
    \begin{innerlist}
        \item PHYS~221 and 222: Laboratory Physics for Science and Engineering.
        \begin{innerlist}
            \item Undergraduate course in introductory laboratory
            physics for science and engineering majors.
            \item I worked as a teaching assistant of two sections
            ($\sim$ 50 students) per semester during this period. I was
            responsible for engaging the student in the laboratory activities and
            grading their assignments (weekly experiment reports and tests). I
            also served as a tutor (2 hrs/week) helping undergraduate students with
            their classwork.
            \item Laboratory Physicists:
            \href{http://relativity.phy.olemiss.edu/~thomas/}{Thomas Jamerson}.
        \end{innerlist}
    \end{innerlist}

\end{outerlist}

\section{Professional Service}

\textbf{Referee} \\
I have served as a referee for the following scientific journals [where ($n$)
indicates the number of refereed manuscripts]:
\blankline
\vspace{0.25cm}
\begin{innerlist}
    \item \emph{Physical Review D} \dotfill\, (12),
    \item \emph{International Journal of Modern Physics D} \dotfill\, (3),
    \item \emph{General Relativity and Gravitation} \dotfill\, (2),
    \item \emph{Physical Review Letters} \dotfill\, (2),
    \item \emph{Universe} \dotfill\, (2),
    \item \emph{The Astrophysical Journal} \dotfill\, (1),
    \item \emph{Classical and Quantum Gravity} \dotfill\, (1),
    \item \emph{European Journal of Physics C} \dotfill\, (1),
    \item \emph{International Journal of Modern Physics E} \dotfill\, (1),
    \item \emph{Modern Physics Letters A} \dotfill\, (1),
    \item \emph{Monthly Notices of the Royal Astronomical Society} \dotfill\, (1),
    \item \emph{Physical Review E} \dotfill\, (1),
    \item \emph{Physics} \dotfill\, (1),
    \item \emph{Particles} \dotfill\, (1),
\end{innerlist}

\blankline

\textbf{Judging committees} \\
I participated in the following undergraduate thesis defense:
\restartlist{bibenum}
\begin{bibenum}
    \item {Tan\'isia de F\'atima de Moraes Cardoso}, \hfill{2010} \\
        Undergraduate student in Physics, Federal University of Par\'a, \\
        Title: \emph{Path integrals in quantum mechanics}.
\end{bibenum}

\vspace{0.1in}

% \section{Professional Experience}

% \href{http://www.asu.edu/}{\textbf{Arizona State University}},
% Tempe, AZ
% \begin{outerlist}

%     \item[] \textit{Assistant Professor}%
%             \hfill \textbf{August 2015 (upcoming)}
%             \begin{innerlist}
%                 \item Joint Appointment:
%                     \begin{innerlist}
%                         \item School of Computing, Informatics, and Decision Systems Engineering
%                         \item School of Sustainability
%                     \end{innerlist}

%                 \item Graduate faculty in Industrial Engineering/Operations
%                     Research, Sustainability, and Animal Behavior.

%                 \item Interdisciplinary laboratory focus on decision
%                     making and organization.
%             \end{innerlist}

%     \item[] \textit{Associate Research Scientist}%
%             \hfill \textbf{August 2014 to present}\\
%         \textit{Postdoctoral Scholar}%
%             \hfill \textbf{July 2012 to August 2014}
%             \begin{innerlist}
%                 \item Supervisor:
%                         \href{http://www.public.asu.edu/~spratt1/}%
%                              {Professor Stephen C.~Pratt}

%                 \item Novel application of sophisticated quantitative
%                     analysis and modeling techniques to animals, with
%                     social insects as a particular focus.

%                 \item Development of new algorithms for robotics and
%                     other autonomous systems based on animal behavior,
%                     with focus on distributed decision making.

%                 \item Supervision of graduate and undergraduate students
%                     in engineering, computer science, and biology in
%                     tasks related to biological analysis and modeling as
%                     well as technological bio\-/mimetic design.
%             \end{innerlist}

% \end{outerlist}

% \halfblankline

% \href{http://www.osu.edu/}{\textbf{The Ohio State University}},
% Columbus, OH
% \begin{outerlist}

%     \item[] \textit{Postdoctoral Researcher}%
%             \hfill \textbf{September 2010 to June 2012}
%             \begin{innerlist}
%                 \item Funding:~\href{http://www.nfs.gov/}{National Science Foundation} Cyber\-/Physical Systems (ENG, \href{http://www.nsf.gov/div/index.jsp?div=eccs}{ECCS})
%                 \begin{innerlist}
%                     \item[$-$] ``Autonomous Driving in Mixed\-/Traffic Urban Environments''
%                         (grant~\href{http://www.nsf.gov/awardsearch/showAward.do?AwardNumber=0931669}{\#0931669})
%                     \item[$-$] Supervisor (co\-/PI):
%                         \href{http://www.cse.ohio-state.edu/~paolo/}%
%                              {Professor Paolo A.~G.~Sivilotti}
%                     \item[$-$] PI:
%                         \href{http://www.ece.ohio-state.edu/~umit/}%
%                              {Professor \"{U}mit \"{O}zg\"{u}ner}
%                 \end{innerlist}

%                 \item Development of new approaches to software
%                     verification in the context of hybrid\-/state and
%                     hybrid\-/time dynamical systems.

%                 \item Supervision of student design project for
%                     novel vehicle\-/to\-/vehicle communications
%                     systems to assist in adaptive cruise control.
%             \end{innerlist}

% \end{outerlist}

% \halfblankline

% \href{http://www.ni.com/}{\textbf{National Instruments}},
% Austin, TX
% \begin{outerlist}

% \item[] \textit{Hardware R\&D Intern for Multifunction DAQ}%
%         \hfill \textbf{June 2003 to September 2003}
% \begin{innerlist}
% \item Designed final verification test fixture for use with STC2 MIO
%         products.
% \item Designed and executed study of the effect of varying burn\-/in time
%         on long\-/term drift of common industry voltage references.
% \end{innerlist}

% \item[] \textit{Hardware R\&D Intern for Multifunction DAQ}%
%         \hfill \textbf{June 2002 to September 2002}
% \begin{innerlist}
% \item Designed and performed validation tests for 16-bit 800 kHz
%         NI-6120 SMIO DAQ.

% \item Designed high\-/quality source to use with NI-5411 arbitrary
%         function generator.
% \end{innerlist}

% \end{outerlist}

% \halfblankline

% \textbf{\href{http://www.ibm.com/}{IBM} Network Storage},
% Research Triangle Park, NC
% \begin{outerlist}

% \item[] \textit{Core Systems Software Developer for FlexNAS}%
%         \hfill \textbf{June 2001 to September 2001}
% \begin{innerlist}
% \item Designed and implemented highly available multihop communications
%         subsystem.
% \item Participated in software development of various vital box
%         services.
% \end{innerlist}

% \end{outerlist}

% \halfblankline

% \href{http://www.calltech.com/}{\textbf{CallTech Communications}},
% Columbus, OH
% \begin{outerlist}

% \item[] \textit{Information Technology Systems Engineer}%
%         \hfill \textbf{June 1997 to May 2001}
% \begin{innerlist}
% \item Responsible for the acquisition, setup, and administration of all
%         hardware and software systems supporting
%         \href{http://www.netwalk.com/}{NetWalk} Internet service and web
%         presence provider.
% \item Designed and implemented state\-/of\-/the\-/art open\-/source
%         highly available load\-/balancing system supporting thousands of
%         virtual servers.
% \item Developed call\-/center software for clients such as
%         CompuServe, AOL, and Priceline.
% \end{innerlist}

% \end{outerlist}

% \halfblankline

% \textbf{MegaLinx Communications}, Dublin, OH
% \begin{outerlist}

% \item[] \textit{Web Developer and Support Representative}%
%         \hfill \textbf{June 1995 to May 1997}
% \begin{innerlist}
% \item Produced web content for commercial clients.
% \item Assisted in administration of UltraSPARC, x86, 680x0, and PowerPC
%         systems.
% \item Developed multi\-/platform open\-/source file\-/sharing solution.
% \item Provided technical support for Internet and web presence
%         customers.
% \end{innerlist}

% \end{outerlist}

\section{Professional Memberships}

I am a member of the following organizations:
%
\begin{innerlist}
\item \href{http://www.sbfisica.org.br}{Sociedade Brasileira de F\'isica}.
\item \href{https://www.aps.org}{American Physical Society}.
    \begin{innerlist}
    \item \href{https://www.aps.org/units/dgrav/}{Division of Gravity}.
    \end{innerlist}
\item \href{http://www.sigmapisigma.org}{Sigma Pi Sigma}.
\end{innerlist}

\vspace{0.1in}

\section{Outreach and Service}

\href{https://twitter.com/arXiv_BR}
{\textbf{arXiv.BR}}
\hfill{Since 2014}

\begin{innerlist}
    \item I maintain and run the Twitter account arxiv\_br since 2014. The
    account shares papers by Brazilian scientists (and foreigners at Brazilian institutions)
    which have been posted to the arXiv the gr-qc, astro-ph and hep-th sections.
    %
    I also regularly post threads on the history of science in Brazil and the latest events
    in the field of gravitational physics in Portuguese.
    %
    As of February 2020, the account has 817 followers, many of whom are lay people and physics
    students.
    %
    The account regularly interacts with the Twitter accounts of major Brazilian institutions such
    as the \href{http://portal.cbpf.br/pt-br/en}{Centro Brasileiro de Pesquisas F\'isicas}
    and the \href{http://www.abc.org.br/}{Acad\^emia Brasileira de Ci\^encias}.
\end{innerlist}

\halfblankline

\href{https://hosilva.github.io/memoria/}{\textbf{Mem\'oria da F\'isica Brasileira}}
\hfill{Since March 2019}

\begin{innerlist}
    \item I maintain and run the website \emph{Mem\'oria da F\'isica Brasileira} (Memory of the Brazilian
    Physics).
    %
    In this website I write biographies of Brazilian physicists and accounts on events in
    the history of physics in Brazil.
    %
    I also keep a database of material related on this topic.
    %
    Currently (March 2019) the project is in its early stages. The goal is to ultimately have a website
    in the style of the \href{http://www-history.mcs.st-and.ac.uk/}{MacTutor History of Mathematics archive}
    but focused on Brazilian physics.
\end{innerlist}

\halfblankline

\href{www.ufpa.br}{\textbf{Federal University of Par\'a}}
\hfill {2009 -- 2011}\\
\textbf{Student representative}

\begin{innerlist}
    \item During the academic years of 2009--2010 and 2010--2011, I was one of the
    two student representatives at the graduate program in physics at the
    Federal University of Par\'a. My role was to act as an active voice representing the
    students and participating in decisions concerning the graduate program.
\end{innerlist}

\halfblankline

\textbf{2015 MS Region VII Science Fair}
\hfill {March 2015}

\begin{innerlist}
    \item Judge of the science fair projects in the ``Lower Fair'' session.
\end{innerlist}

\halfblankline

\href{www.olemiss.edu}{\textbf{University of Mississippi}},
        \hfill {April 2015 -- January 2016}\\
            \href{http://dos.orgsync.com/org/pgsa}{Physics Graduate Student Association (PGSA)}

\begin{innerlist}
    \item I am a founding member of the PGSA at University of Mississippi.
    PGSA's goal is to estimulate the interaction among the physics student, through
    events and social activities. During the period from April 2015 to January 2016
    I served as vice-president.
\end{innerlist}

\vspace{0.1in}

\section{Organization}

\href{http://www.ufpa.br/ppgf/ippgf.html}
{\textbf{1st School of the Graduate Program in Physics \\ of the Federal University of Par\'a}},
\hfill{January 2011}

\begin{innerlist}
    \item Co-organized with Profs. Petrus A. A. J\'unior,
    Danilo T. Alves and Lu\'is C. B. Crispino. This week-long school offered
    mini-courses, at the advanced undergraduate level, on Electromagnetism,
    Statistical Mechanics and Quantum Mechanics (and seminars on research
    topics by the faculty of Federal University of Par\'a) for
    university level students in the Bel\'em area, attracting participants
    from four universities.
\end{innerlist}

\halfblankline

\href{http://dos.orgsync.com/org/pgsa/ResearchSymposium2015}
{\textbf{Physics Graduate Student Research Symposium}},
\hfill{September 2015}

\begin{innerlist}
    \item Co-organized with other PGSA members. The event consisted of an afternoon
    with short communications by the student body of the Department of Physics and
    Astronomy, at University of Mississippi.
\end{innerlist}

\halfblankline

\href{http://dos.orgsync.com/org/pgsa/UM_MSU_SYMP}
{\textbf{1st Annual UM-MSU Joint Physics Research Symposium}},
\hfill{February 2016}

\begin{innerlist}
    \item Co-organized with other PGSA members.
    The symposium consisted of a day with short communications (and poster presentations)
    by undergraduate and graduate students from the Departments of Physics
    and Astronomy from both University of Mississippi and Mississippi State
    University.
\end{innerlist}

% \vspace{0.1in}

% \section{Outreach \& Acitivities}

% \href{http://www.olemiss.edu/}{\textbf{University of Mississippi}},
% Oxford, MS, USA
% \begin{outerlist}

% \item[] \textit{Grader}%
%         \hfill \textbf{August to December 2015}
% \begin{innerlist}
% \item Main instructor: James Hill
% \item Graded assignments for introductory astronomy class (ASTR~103).
% \end{innerlist}

% \item[] \textit{Teaching Assistant} \hfill \textbf{August 2012 to July 2014}
%     \begin{innerlist}
%         \item PHYS~221 and 222: Laboratory Physics for Science and Engineering
%         \begin{innerlist}
%             \item Undergraduate course in introductory physics for science and
%             engineering majors
%             \item I served as a teaching assistant of two sections
%             ($\sim$ 50 students) per semester during this period. I was
%             responsible engaging the student in the laboratory activities and
%             grading their assingments (experiment reports and tests)
%             \item Laboratory Physicists: Thomas Jamerson
%         \end{innerlist}
%     \end{innerlist}

% \end{outerlist}

\vspace{0.1in}

\section{Computational Skills}

Computer programming and others:
%
\begin{innerlist}
    \item Python, C++, Gnuplot, \LaTeX, HTML, CSS and Julia.
\end{innerlist}

\halfblankline

Mathematical software:
%
\begin{innerlist}
    \item Mathematica and Maple.
\end{innerlist}

\halfblankline

Operating systems:
%
\begin{innerlist}
    \item Microsoft Windows, Mac OSX and Linux.
\end{innerlist}

\blankline

\section{Languages}

Portuguese (Native), English (Fluent) and Spanish (Basic).

% \section{References Available to Contact}

% \href
% {http://www.public.asu.edu/~spratt1/}
% {\textbf{Dr.~Stephen C.~Pratt}}
% (e\-/mail:~\href{mailto:stephen.pratt@asu.edu}{stephen.pratt@asu.edu}; phone:~+1-480-727-9425)
% %
% \begin{innerlist}
%     \item Associate Professor,
%         \href{http://sols.asu.edu/}{School of Life Sciences},
%         \href{http://www.asu.edu/}{Arizona State University}

%     \item[$\diamond$] School of Life Sciences, PO Box 874501, Tempe, AZ
%         85287-4501

%     \item[$\star$] \emph{Dr.~Pratt is my current postdoctoral supervisor.}
% \end{innerlist}

% \halfblankline

% \textbf{Dr.~Spring M.~Berman}
% (e\-/mail:~\href{mailto:Spring.Berman@asu.edu}{Spring.Berman@asu.edu}; phone:~+1-480-965-4431)
% %
% \begin{innerlist}
%     \item Assistant Professor,
%         \href{http://sols.asu.edu/}{Mechanical and Aerospace Engineering},
%         \href{http://www.asu.edu/}{Arizona State University}

%     \item[$\diamond$] School for Engineering of Matter, Transport, and
%         Energy, PO Box 876106, Tempe, AZ
%         85287-6106

%     \item[$\star$] \emph{Dr.~Berman is collaborator on my bio\-/mimicry work.}
% \end{innerlist}

% \halfblankline

% \href
% {http://cosmos.asu.edu/}
% {\textbf{Dr.~Paul C.~W.~Davies}}
% (e\-/mail:~\href{mailto:Paul.Davies@asu.edu}{Paul.Davies@asu.edu}; phone:~+1-480-965-3240)
% %
% \begin{innerlist}
%     \item Regents Professor and Director,
%         \href{http://beyond.asu.edu/}{Beyond Center for Fundamental Concepts in Science},
%         \href{http://www.asu.edu/}{Arizona State University}

%     \item[$\diamond$] Beyond Center for Fundamental Concepts in Science,
%         P.O. Box 871504, Tempe, AZ
%         85287-1504

%     \item[$\star$] \emph{Dr.~Davies is collaborator on my
%         origins\-/of\-/life work.}
% \end{innerlist}

% \halfblankline

% \href
% {http://emergence.asu.edu/}
% {\textbf{Dr.~Sara Imari Walker}}
% (e\-/mail:~\href{mailto:sara.i.walker@asu.edu}{sara.i.walker@asu.edu}; phone:~+1-480-727-2394)
% %
% \begin{innerlist}
%     \item Assistant Professor,
%         \href{http://sese.asu.edu/}{School of Earth and Space Exploration},
%         \href{http://www.asu.edu/}{Arizona State University}

%     \item[$\diamond$] ASU School of Earth and Space Exploration,
%         PO Box 871404, Tempe, AZ
%         85287-1404

%     \item[$\star$] \emph{Dr.~Walker is collaborator on my
%         origins\-/of\-/life work.}
% \end{innerlist}

% \halfblankline

% \textbf{Dr.~Pietro Michelucci}
% (e\-/mail:~\href{mailto:pem@thinksplash.com}{pem@thinksplash.com}; phone: +1-571-235-3288)
% \begin{innerlist}
%     \item Principal,
%         ThinkSplash LLC, Washington, DC

%     \item[$\star$] \emph{I co\-/authored a chapter in the \emph{Handbook
%         of Human Computation}, for which Dr.~Michelucci was the
%         editor\-/in\-/chief.}
% \end{innerlist}

% \halfblankline

% \href
% {http://www.cse.ohio-state.edu/~paolo/}
% {\textbf{Dr.~Paolo A.~G.~Sivilotti}}
% (e\-/mail:~\href{mailto:sivilotti.1@osu.edu}{sivilotti.1@osu.edu}; phone: +1-614-292-5835)
% \begin{innerlist}
%     \item Associate Professor,
%         \href{http://www.cse.ohio-state.edu/}{Computer Science and Engineering},
%         \href{http://www.osu.edu/}{The Ohio State University}

%     \item[$\diamond$] 395 Dreese Laboratories, 2015 Neil Ave., Columbus,
%         OH  43210

%     \item[$\star$] \emph{Dr.~Sivilotti is my past postdoctoral
%         supervisor.}
% \end{innerlist}

% \halfblankline

% \href
% {http://www.cse.ohio-state.edu/~weide/}
% {\textbf{Dr.~Bruce W.~Weide}}
% (e\-/mail:~\href{mailto:weide.1@osu.edu}{weide.1@osu.edu}; phone:~+1-614-292-1517)
% \begin{innerlist}
%     \item Professor and Associate Chair,
%         \href{http://www.cse.ohio-state.edu/}{Computer Science and
%         Engineering}\\
%         \href{http://www.osu.edu/}{The Ohio State University}

%     \item[$\diamond$] 395 Dreese Laboratories, 2015 Neil Ave., Columbus,
%         OH  43210

%     \item[$\star$] \emph{Dr.~Weide is a co\-/PI on the NSF grant that
%         funded my previous postdoctoral position.}
% \end{innerlist}

% \halfblankline

% \href
% {http://hamilton-lab.wikidot.com/}
% {\textbf{Dr.~Ian M.~Hamilton}}
% (e\-/mail:~\href{mailto:hamilton.598@osu.edu}{hamilton.598@osu.edu}; phone:~+1-614-292-9147)
% %
% \begin{innerlist}
%     \item Assistant Professor,
%         \href{http://eeob.osu.edu/}{Evolution, Ecology, and Organismal Biology}
%         and
%         \href{http://www.math.ohio-state.edu/}{Mathematics}\\
%         \href{http://www.osu.edu/}{The Ohio State University}

%     \item[$\diamond$] 300 Aronoff Laboratory, 318 W.~12th Avenue,
%         Columbus, OH  43210

%     \item[$\star$] \emph{Dr.~Hamilton has been a valuable
%         interdisciplinary resource to me.}
% \end{innerlist}

% \halfblankline

% \href
% {http://www.ece.osu.edu/~passino/}
% {\textbf{Dr.~Kevin M.~Passino}}
% (e\-/mail:~\href{mailto:passino.1@osu.edu}{passino.1@osu.edu}; phone:~+1-614-312-2472)
% %
% \begin{innerlist}
%     \item Professor,
%         \href{http://www.ece.osu.edu/}{Electrical and Computer
%         Engineering},
%         \href{http://www.osu.edu/}{The Ohio State University}

%     \item[$\diamond$] 205 Dreese Laboratories, 2015 Neil Ave., Columbus,
%         OH  43210

%     \item[$\star$] \emph{Dr.~Passino was my graduate adviser.}
% \end{innerlist}

% \halfblankline

% \href
% {http://www.ece.osu.edu/~serrani/}
% {\textbf{Dr.~Andrea Serrani}}
% (e\-/mail:~\href{mailto:serrani.1@osu.edu}{serrani.1@osu.edu}; phone:~+1-614-292-4976)
% %
% \begin{innerlist}
%     \item Associate Professor,
%         \href{http://www.ece.osu.edu/}{Electrical and Computer Engineering}\\
%         \href{http://www.osu.edu/}{The Ohio State University}

%     \item[$\diamond$] 205 Dreese Laboratories, 2015 Neil Ave., Columbus,
%         OH  43210

%     \item[$\star$] \emph{Dr.~Serrani was a member of my doctoral
%         committee.}
% \end{innerlist}

% \halfblankline

% \href
% {http://feh.osu.edu/staff/view.html?UID=798}
% {\textbf{Dr.~Richard J.~Freuler}}
% (e\-/mail:~\href{mailto:freuler.1@osu.edu}{freuler.1@osu.edu}; phone: +1-614-688-0499)
% \begin{innerlist}
%     \item Professor of Practice,
%         \href{http://mae.osu.edu/}{Mechanical and Aerospace Engineering}\\
%         \href{http://www.osu.edu/}{The Ohio State University}

%     \item[$\diamond$] 244 Hitchcock Hall, 2070 Neil Ave., Columbus, OH  43210

%     \item[$\star$] \emph{Dr.~Freuler coordinates the Fundamentals of
%         Engineering for Honors program in which I served as an
%         instructor early in my academic career.}
% \end{innerlist}

% \halfblankline

% \href
% {http://mae.osu.edu/people/staab.1}
% {\textbf{Dr.~George H.~Staab}}
% (e\-/mail:~\href{mailto:staab.1@osu.edu}{staab.1@osu.edu}; phone: +1-614-292-7920)
% \begin{innerlist}
%     \item Associate Professor,
%         \href{http://mae.osu.edu/}{Mechanical and Aerospace Engineering}\\
%         \href{http://www.osu.edu/}{The Ohio State University}

%     \item[$\diamond$] W192 Scott Laboratory, 201 W.~19th Ave., Columbus, OH  43210

%     \item[$\star$] \emph{Dr.~Staab is the faculty adviser for the OSU
%         FIRST robotics and engineering outreach group of which I was a
%         four\-/year member and team leader.}
% \end{innerlist}

% \halfblankline

% \textbf{Dr.~Clayton Daigle}
% (e\-/mail:~\href{mailto:Clayton.Daigle@silabs.com}{Clayton.Daigle@silabs.com}; phone: +1-512-532-5935)
% \begin{innerlist}
%     \item Mixed-Signal Engineer,
%         \href{http://www.silabs.com/}{Silicon Laboratories}, Austin, TX

%     \item[$\star$] \emph{Dr.~Daigle was my direct supervisor when I
%         worked for National Instruments as an analog hardware R\&D
%         engineer.}
% \end{innerlist}

% The ``More Info'' section may not be necessary; make sure it's short
% so it doesn't prevent people from seeing references available to
% contact.

\blankline

\section{More \\ Information}

More information can be found in my website \\%
\url{www.phy.olemiss.edu/~hosilva}.

\end{document}

%%%%%%%%%%%%%%%%%%%%%%%%%% End CV Document %%%%%%%%%%%%%%%%%%%%%%%%%%%%%

%----------------------------------------------------------------------%
% The following is copyright and licensing information for
% redistribution of this LaTeX source code; it also includes a liability
% statement. If this source code is not being redistributed to others,
% it may be omitted. It has no effect on the function of the above code.
%----------------------------------------------------------------------%
% Copyright (c) 2007, 2008, 2009, 2010, 2011 by Theodore P. Pavlic
%
% Unless otherwise expressly stated, this work is licensed under the
% Creative Commons Attribution-Noncommercial 3.0 United States License. To
% view a copy of this license, visit
% http://creativecommons.org/licenses/by-nc/3.0/us/ or send a letter to
% Creative Commons, 171 Second Street, Suite 300, San Francisco,
% California, 94105, USA.
%
% THE SOFTWARE IS PROVIDED "AS IS", WITHOUT WARRANTY OF ANY KIND, EXPRESS
% OR IMPLIED, INCLUDING BUT NOT LIMITED TO THE WARRANTIES OF
% MERCHANTABILITY, FITNESS FOR A PARTICULAR PURPOSE AND NONINFRINGEMENT.
% IN NO EVENT SHALL THE AUTHORS OR COPYRIGHT HOLDERS BE LIABLE FOR ANY
% CLAIM, DAMAGES OR OTHER LIABILITY, WHETHER IN AN ACTION OF CONTRACT,
% TORT OR OTHERWISE, ARISING FROM, OUT OF OR IN CONNECTION WITH THE
% SOFTWARE OR THE USE OR OTHER DEALINGS IN THE SOFTWARE.
%----------------------------------------------------------------------%
